\documentclass{article}
\usepackage[utf8]{inputenc}
\usepackage{fullpage,enumitem,amssymb,amsmath,xcolor,graphicx,multicol}
\usepackage[most]{tcolorbox}

\title{\textbf{API Workshop Notes}}
\author{Maria Berova}
\date{\today}

\makeatletter
\renewcommand*\env@matrix[1][*\c@MaxMatrixCols c]{%
  \hskip -\arraycolsep
  \let\@ifnextchar\new@ifnextchar
  \array{#1}}
\makeatother

\newcommand{\sub}{\subsection*}
\newcommand{\im}{\text{Im}}
\def\vvec#1#2#3#4{\begin{pmatrix} #1 \\ #2 \\ #3 \\ #4 \end{pmatrix}}
\newtcolorbox{mybox}[2][]{colback=black!5!white,
colframe=black!75!black,fonttitle=\bfseries,
colbacktitle=black!85!black,enhanced,
attach boxed title to top left={yshift=-2mm},
title={#2}}

\begin{document}
\maketitle

An API wraps together one or more endpoints

\begin{itemize}
  \item API is like a class
  \item Endpoints are called with parameters and return a value (in JSON or XML)
\end{itemize}

Tokens

\begin{itemize}
  \item Used to control and monitor API access
  \item API requires a token parameter -- will only return data on valid token
  \item Generate token with username/password
\end{itemize}

CORS

\begin{itemize}
  \item Way of controlling who gets to call the API -- what platform. E.g., websites, servers, mobile app
  \item Can whitelist a specific domain (like inrix.com)
  \item For us the easiest way is to just allow all access
\end{itemize}

INRIX forces us to use CORS so that no one else can access the tokens for API. So we have to use CORS to access data.

\begin{itemize}
  \item Webapp requests Proxy API for token
  \item Proxy sends username/pass to INRIX 
  \item INRIX returns token to proxy, which returns to webapp
  \item Webapp uses token to access data from INRIX
\end{itemize}

%% https://github.com/INRIX-Aashay-Motiwala/INRIX_Hack_Client_Server_Demo

\end{document}